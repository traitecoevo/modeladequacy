\documentclass[a4paper,12pt]{article}

\setlength{\parskip}{1ex plus1pt} % threequarters
\RequirePackage[hmargin=3.5cm,vmargin=3cm]{geometry}
\usepackage{graphicx}
\pagestyle{empty}
\usepackage{MnSymbol}
\usepackage[mathlf,textlf,minionint]{MinionPro}
\usepackage[T1]{fontenc}
\usepackage{textcomp}
\usepackage[numbers, sort&compress]{natbib}

\begin{document}

%\begin{flushleft}
%\textsc{Referenced Abstract} for ``The adequacy of phylogenetic trait models''\\[0.5cm]
%M.W. Pennell, R.G. FitzJohn, W.K. Cornwell \& L.J. Harmon
%\end{flushleft}

\section*{Background}

Phylogenetic comparative analyses are now routinely applied to test evolutionary hypotheses in a variety of fields \citep{PennellHarmon}. Such analyses require the use of a statistical model of trait evolution. Over the last decade there has been an influx of statistical approaches for modeling trait evolution on phylogenies (e.g. \citep{ButlerKing2004, Omeara2006, Eastman2011, Beaulieu2012}). While we can compare the \textit{relative} fit of alternative models \citep{Harmon2010}, we do not have any procedure for examining the \textit{absolute} fit of a model, that is, measuring how adequate our models are. Using an inadequate model has serious consequences regarding the inferences we can draw from the analysis \citep{Boettiger2012, SlaterPennell}. Here we present a unified and general statistical framework for assessing the adequacy of trait models.

\section*{Methodology/Principal findings}

Our framework is based on the statistical properties of phylogenetic independent contrasts \citep{Felsenstein1985}. To assess model adequacy, we rescale the branch lengths of a phylogeny according to model parameters such that if the model is indeed adequate, the contrasts will be independent and identically distributed; we term such a phylogeny a ``unit tree''. We calculate a set of test statistics on the contrasts computed on the unit tree. We then simulate a large number of datasets under the appropriate model, calculate the test statistics on the contrasts of the simulated data and compare the observed to simulated summary statistics. This allows us to assess whether a model is adequate. If the model is inadequate, our summary statistics provide insight into how the data deviate from the model. We apply our approach to a number of large comparative datasets of plant functional traits collected from the literature using a recently published megaphylogeny of Angiosperms \citep{Zanne2013}. We demonstrate that commonly applied models are seriously inadequate. Furthermore, model inadequacy scales very tightly with tree size, suggesting that simple models are especially poor fits to data on megaphylogenies.

\section*{Conclusions/Significance}

Our approach is generally applicable to almost all comparative analyses of continuous traits. The results of our study raise substantial questions regarding the interpretations of a large number of comparative studies. We suggest that the best way forward is to couple conventional model comparison with an assessment of model adequacy. This will improve the reliability of inferences drawn from comparative analyses. We believe that tests for model adequacy have the potential to fundamentally change how researchers use phylogenies to ask evolutionary questions. 

\bibliographystyle{plos}
\bibliography{plos-presub}

\end{document}
